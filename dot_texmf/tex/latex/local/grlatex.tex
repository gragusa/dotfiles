%% Documentation for grlatex.cls
\documentclass{ltxdoc}
\usepackage{booktabs}
\usepackage{xcolor}
\usepackage{hyperref}
\usepackage{listings}
\usepackage{enumitem}

\title{The \textsf{grlatex} document class}
\author{Giuseppe Ragusa}
\date{\today}

\lstset{
  basicstyle=\ttfamily\small,
  commentstyle=\color{gray},
  keywordstyle=\color{blue},
  stringstyle=\color{purple},
  breaklines=true,
  breakatwhitespace=true,
  frame=single,
  language=[LaTeX]TeX,
  texcsstyle=*\color{blue},
  moretexcs={RequirePackage,DeclareOption,ProcessOptions,LoadClass,newcommand,renewcommand,ifx,nameuse}
}

\begin{document}

\maketitle

\tableofcontents

\section{Introduction}

The \textsf{grlatex} class provides a specialized environment for creating academic documents based on the standard \textsf{article} class. It's designed to offer enhanced typography, customizable bibliography systems, and other features useful for academic writing.

\section{Usage}

To use the class, simply specify it in the \verb|\documentclass| command:

\begin{lstlisting}
\documentclass[<options>]{grlatex}
\end{lstlisting}

Where \verb|<options>| are one or more of the available options described in Section~\ref{sec:options}.

\section{Class Options}
\label{sec:options}

The \textsf{grlatex} class accepts all options available to the standard \textsf{article} class, plus several additional options specific to \textsf{grlatex}.

\subsection{Paper and Layout Options}

These options are passed directly to the \textsf{article} class:

\begin{description}
\item[a4paper] Set the paper size to A4 (210 × 297 mm).
\item[letterpaper] Set the paper size to US Letter (8.5 × 11 inches).
\item[legalpaper] Set the paper size to US Legal (8.5 × 14 inches).
\item[executivepaper] Set the paper size to Executive (7.25 × 10.5 inches).
\item[landscape] Set the page orientation to landscape.
\item[10pt, 11pt, 12pt] Set the base font size (default is 10pt).
\item[oneside, twoside] Specify whether the document will be printed on one or both sides of the paper.
\item[draft, final] In draft mode, overfull boxes are marked with a black box in the margin.
\item[titlepage, notitlepage] Specify whether the title should be on a separate page.
\item[openright, openany] Determine where chapters begin (relevant for book-like documents).
\item[onecolumn, twocolumn] Format the document in one or two columns.
\item[openbib] Use the open bibliography format.
\end{description}

\subsection{GR-Specific Options}

The following options are specific to the \textsf{grlatex} class:

\begin{description}
\item[biblatex] \emph{(Default)} Use the \textsf{biblatex} package for bibliography management with the following settings:
  \begin{itemize}
    \item style=authoryear
    \item backend=biber
    \item giveninits=true
    \item natbib=true
    \item date=year
    \item uniquelist=false
    \item uniquename=init
    \item isbn=false
    \item maxcitenames=3
    \item dashed=false
    \item maxbibnames=999
    \item doi=true
    \item backref=true
  \end{itemize}

\item[bibtex] Use the \textsf{natbib} package with the \texttt{authoryear} and \texttt{round} options instead of \textsf{biblatex}.

\item[coloredheadings] Use colored section headings.

\item[plainheadings] Use plain (non-colored) section headings.
\end{description}

\section{Required Packages}

The \textsf{grlatex} class automatically loads the following packages:

\begin{table}[ht]
\centering
\begin{tabular}{ll}
\toprule
\textbf{Package} & \textbf{Purpose} \\
\midrule
fontenc (T1) & Enhanced font encoding \\
lmodern & Latin Modern fonts \\
amsmath & Mathematical typesetting \\
amssymb & Mathematical symbols \\
amsthm & Theorem environments \\
appendix & Enhanced appendix formatting \\
bm & Bold mathematical symbols \\
booktabs & Enhanced table formatting \\
biblatex/natbib & Bibliography management (depending on option) \\
\bottomrule
\end{tabular}
\caption{Packages loaded by \textsf{grlatex}}
\end{table}

\section{Examples}

\subsection{Basic Document}

\begin{lstlisting}
\documentclass{grlatex}

\title{My Document}
\author{Author Name}
\date{\today}

\begin{document}

\maketitle

\section{Introduction}
This is an example document using the grlatex class.

\end{document}
\end{lstlisting}

\subsection{Document with Bibliography}

Using biblatex (default):

\begin{lstlisting}
\documentclass{grlatex}

\title{Document with References}
\author{Author Name}

\begin{document}

\maketitle

\section{Introduction}
This document cites an important paper \citep{Smith2020}.

\printbibliography

\end{document}
\end{lstlisting}

Using bibtex with natbib:

\begin{lstlisting}
\documentclass[bibtex]{grlatex}

\title{Document with References}
\author{Author Name}

\begin{document}

\maketitle

\section{Introduction}
This document cites an important paper \citep{Smith2020}.

\bibliographystyle{plainnat}
\bibliography{mybibfile}

\end{document}
\end{lstlisting}

\section{Implementation Notes}

The class is based on the standard \textsf{article} class and modifies it to provide enhanced functionality. The bibliography system selection is implemented as follows:

\begin{lstlisting}
\newcommand{\grlatex@bibengine}{biblatex} % Default to biblatex

\DeclareOption{biblatex}{\renewcommand{\grlatex@bibengine}{biblatex}}
\DeclareOption{bibtex}{\renewcommand{\grlatex@bibengine}{bibtex}}

% Later in the code:
\ifx\grlatex@bibengine\@nameuse{biblatex}
  \RequirePackage[style=authoryear, ...]{biblatex}
\else
  \RequirePackage[authoryear,round]{natbib}
\fi
\end{lstlisting}

\appendix
\section{Implementation}

The full implementation of the class is not shown here. Please refer to the source file \texttt{grlatex.cls} for details.

\section{Version History}

\begin{description}
\item[v1.0] Initial release.
\end{description}

\end{document}
